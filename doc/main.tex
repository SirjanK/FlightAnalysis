\documentclass{article}
\usepackage[final]{nips}
\usepackage[utf8]{inputenc} % allow utf-8 input
\usepackage[T1]{fontenc}    % use 8-bit T1 fonts
\usepackage[english]{babel}
\usepackage{amsthm}
\usepackage{hyperref}       % hyperlinks
\usepackage{url}            % simple URL typesetting
\usepackage{booktabs}       % professional-quality tables
\usepackage{amsfonts}       % blackboard math symbols
\usepackage{nicefrac}       % compact symbols for 1/2, etc.
\usepackage{microtype}      % microtypography
\usepackage{tikz}           % plotting
\usepackage{pgfplots}
\pgfplotsset{compat=1.12}
\usepackage{amsmath,textcomp,amssymb,geometry,graphicx,enumerate}
\usepackage{algorithm} % Boxes/formatting around algorithms
\usepackage[noend]{algpseudocode} % Algorithms

\newtheorem{theorem}{Theorem}[section]
\newtheorem{corollary}{Corollary}[theorem]
\newtheorem{lemma}[theorem]{Lemma}
\newtheorem{definition}{Defn}[section]

\newcommand\given[1][]{\:#1\vert\:}
\definecolor{mediumpurple}{rgb}{0.58, 0.44, 0.86}

\title{Flight Delay Prediction}

\author{%
  Sirjan Kafle \\
  \texttt{Unsullied by Sponsorship since 1997} \\
  \texttt{sxkafle12@gmail.com} \\
  \texttt{https://flightdelay.us}
}

\begin{document}

\maketitle

\section{Introduction}
We build an app that models the distribution of flight delays conditioned on variables like:
\begin{itemize}
  \item Origin Airport
  \item Destination Airport
  \item Airline
  \item Time of Day
\end{itemize}

More specifically, we want users to provide a subset of those "conditional variables" and we'll output information on the probability that the delay
time will exceed certain thresholds, e.g. 30 minutes, 1 hour, etc along with plotting those probabilities vs delay time.

We are resource constrained based on the container size of the web app we're deploying to, so we describe compression methods for these distributions by modeling them as exponentials - we store
and load the parameters of that distribution for the app rather than full flight data. There are nearly 46M flights in the dataset from 2018 to 2024, so
this is a significant reduction in size. In this doc, we explain a general framework for building algorithms to efficiently model these probabilities at scale.

\section{Modeling Conditional Distributions as an Exponential}
\subsection{Definitions}
Let $D$ be the continuous random variable representing the delay time of a flight. Let $C_i$ be the $i$th conditional variable, e.g. origin airport.
In our cases $C_i$ are categorical, although they don't have to be. Even for time of day, we've divided it into 4 categories with six hour intervals: morning, afternoon, evening, night.
For the purposes of the theory, we'll derive all our results for generic $C_i$, which means these techniques are extendable to other conditioning variables
in the future.

We denote the PDF of the conditional distribution as $f_D(\delta|\{C_i\})$ where $\delta$ is the delay time. What we actually care about, however, are
the cases where $\delta > 0$, so we actually want to model $D' = \max \{D, 0\}$ well.

\subsection{Proposed Model}
Observe that for any $\delta > 0$,
$$f_{D'}(\delta | \{C_i\}) = f_{D'}(\delta, D > 0 | \{C_i\}) = f_{D'}(\delta | \{C_i\}, D > 0)\Pr(D > 0)$$
We model conditional distribution on $D > 0$ as an exponential distribution with parameter $\lambda$.

We model $\Pr(\delta > 0)$ as a Bernoulli distribution with parameter $p$.

Therefore, our final model for $f_{D'}(\delta | \{C_i\})$ is:
$$f_{p, \lambda}(\delta) = \begin{cases}
  0 & \delta < 0 \\
  1 - p & \delta = 0 \\
  p\lambda e^{-\lambda \delta} & \delta > 0
\end{cases}$$
It's easy to see that this is a valid PDF since $\int_0^\infty p\lambda e^{-\lambda \delta} = p$.

\subsection{Maximum Likelihood Estimate of Parameters}
\subsubsection{Derivation}
We want to fit $p$, $\lambda$ to best fit our data.

Take a set of conditional variables $\{C_i=c_i\}$. From our data, assume we have $n$ samples such that the conditionals $\{C_i=c_i\}$ are
satisfied. Denote these delays as $\delta_1, \delta_2, \ldots, \delta_n$.

Let $m = |\{i: \delta_i > 0\}|$ be the number of positive delays.

We make the assumption that the delays are iid. We first derive the log likelihood function of the data after transforming $\delta_i' = \max\{\delta_i, 0\}$.:

\begin{align*}
  h(p, \lambda) = \log\mathcal{L}(p, \lambda ; \{\delta_1', \delta_2', \ldots, \delta_n'\}) &= \log \prod_{i=1}^n f_{p, \lambda}(\delta_i') \\
  &= \sum_{\{i: \delta_i \leq 0\}} \log (1 - p) + \sum_{\{i: \delta_i > 0\}} \log p + \log \lambda - \lambda \delta_i \\
  &= (n - m)\log(1 - p) + m\log p + m\log \lambda - \lambda \sum_{\{i : \delta_i > 0\}} \delta_i
\end{align*}
We want to find:
$$\arg \max_{p, \lambda} \mathcal{L}(p, \lambda ; \{\delta_1', \delta_2', \ldots, \delta_n'\})$$
We compute partial derivatives:
\begin{align*}
  \frac{\partial h}{\partial p} &= -\frac{n - m}{1 - p} + \frac{m}{p} = 0 \\
  \frac{\partial h}{\partial \lambda} &= \frac{m}{\lambda} - \sum_{\{i : \delta_i > 0\}} \delta_i = 0
\end{align*}
Yielding us our estimates:
\begin{align*}
  \hat{p} &= \frac{|\{i: \delta_i > 0\}|}{n} \\
  \hat{\lambda} &= \frac{|\{i: \delta_i > 0\}|}{\sum_{\{i : \delta_i > 0\}} \delta_i}
\end{align*}

\subsection{Delay Survival Function}
Our ultimate goal is to display the survival function or $P(D > \delta | \{C_i\})$ for $\delta > 0$ to user queries. Based on our model we can derive this (for $\delta > 0$):
\begin{align*}
  \Pr(D > \delta) &= \int_\delta^\infty p\lambda e^{-\lambda x}dx \\
  &= -pe^{-\lambda x}\bigg|_\delta^\infty \\
  &= pe^{-\lambda \delta}
\end{align*}

\end{document}